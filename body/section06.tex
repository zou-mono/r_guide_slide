
%%% Local Variables:
%%% mode: latex
%%% TeX-master: t
%%% End:
\section{空间数据绘图系统}

\begin{frame}[c]{\subsecname}{}
      \begin{ornamentblock}
        {我们借助于外感官(我们意识的一种性质)表象给我们自己外面的对象,这些对
象毫无例外的在空间里面。这些对象的形状、大小、以及它们相互间的关系是在
空间里被规定的或能够在空间里被规定的。\\
空间不是一个从外部经验得来的经验概念。因为为使着某种感觉与我以外的某些
东西发生关系,以及同样地为着我能把那些感觉表象为互相在外、互相靠近,从
而不只是彼此不同,并且是在不同的地方,这样就一定要以空间观念为前提。\\
          \rightline{\textemdash《康德·纯粹理性批判》}}
      \end{ornamentblock}
\end{frame}

\subsection{GIS和R}
\begin{frame}[t]{\subsecname}{}
\begin{itemize}
\item<1-> GIS是一组强大的工具集,可以用来收集、存储、任意检索、转换和
显示来自真实世界有特殊用途的\emphText{空间数据}\footnotemark[1]
\end{itemize}

\footnotetext[1]{
Burrough, P. A. and McDonnell, R. A. (1998). \emph{Principles of Geographical
Information Systems}. Oxford University Press, Oxford.}
\end{frame}

\subsection{R的空间数据类}

\subsection{空间数据的导入导出}

\subsection{空间数据可视化}