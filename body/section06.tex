
%%% Local Variables:
%%% mode: latex
%%% TeX-master: t
%%% End:
\section{空间数据绘图系统}

\begin{frame}[c]{\subsecname}{}
      \begin{ornamentblock}
        {我们借助于外感官(我们意识的一种性质)表象给我们自己外面的对象,这些对
象毫无例外的在空间里面。这些对象的形状、大小、以及它们相互间的关系是在
空间里被规定的或能够在空间里被规定的。\\
空间不是一个从外部经验得来的经验概念。因为为使着某种感觉与我以外的某些
东西发生关系,以及同样地为着我能把那些感觉表象为互相在外、互相靠近,从
而不只是彼此不同,并且是在不同的地方,这样就一定要以空间观念为前提。\\
          \rightline{\textemdash《康德·纯粹理性批判》}}
      \end{ornamentblock}
\end{frame}